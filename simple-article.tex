 \documentclass{article}

% any line with a percent symbol is ignored after the symbol


% to increase the line spacing
\renewcommand{\baselinestretch}{1.5} % this is equal to double spacing
% \renewcommand{\baselinestretch}{1.3} % this is equal to 1.5 spacing



\begin{document}

\title{Simple \LaTeX\ Document}
\author{P Sunthar}

\date{\today} % this is the default
\date{10 Sep 2010}

\maketitle

\begin{abstract}
 This article shows the important tags in an \emph{article} class. Article contains
 sections, subsections, and paragraphs as the main content classifications.
\end{abstract}



\section{Introduction}

Hello World!

In latex text file number of a              spaces between words do not count.
New
lines
in 
a 
paragraph
also
does
not
count. This whole section will be treated as a paragraph.

What distinguishes one paragraph from another is a blank line. Like there
is a blank line here.

An important tip is to avoid trying obtain explicit spaces in the compiled
document, like spaces, indents, blank lines (in the final document not
in the plain text file). There are ways to do this, but that is the job
of the editor, who provides the class file,  \emph{not the job of the
content provider} which is you.


\section{Document Dividers}
 The article document can be divided into several sections and subsections.
 The sections can be numbered or not numbered.
 \subsection{First subsection}
  There is nothing in this subsection
 \subsection{Second subsection}
  Subsections can have upto two subsubs like subsection, and subsubsection.
  \subsubsection{This is  under subsect}

  Emphasize depends on the environment it is in.  Normally it is
  \emph{italics},
   but \emph{if the environment is in italics then it is in \emph{Roman}. Like
   in this case}.

 
  \subsubsection{This is  under subsect}

  More about divisions here

 \paragraph{Titled Paragraph}

 By default all the numbers for the sections will be displayed.  The fourth
level of document division is called as a paragraph.  Just as each section
has a title a paragraph can also have a title.
 


\end{document}
Anything below this is ignored
