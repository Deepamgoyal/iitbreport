\documentclass{article}

% \documentclass{book}
% any line with a percent symbol is ignored after the symbol






\begin{document}


% Comment out these three lines to get the minimal LaTeX document
\title{Common Errors in \LaTeX Document}
\author{Sunthar (Change this)}
\maketitle


\begin{abstract}
  This document is written to help you find common errors in a \LaTeX\ Document. 
\end{abstract}


Find the typesetting errors in the following sentences. The errors may
not be captured by your \LaTeX\ programme.
\begin{enumerate}

\item The words \TeX\Latex can be typeset using \verb|\TeX| and
\verb|\LaTeX| commands.


\item Two words have to be separated by atleast one space or one
  newline character (end of line, carriage return). 
                   Any      number       of   spaces
between       two        words      amounts     just one space.
Similarly 
we 
can
have
words
written
in fresh
line
without
blank 
lines
in between.


\item A paragraph is separated by a blank line.  




            Any amount of blank lines is simply equal to one \item.



\item \LaTeX macros are always preceeded by a $\backslash$ symbol.  The
  macros are like inline (or \verb|#define| functions) of the
  C-language.  Some macros accept arguments. Each argument must be
  enclosed in curly braces \{ \}.  Example a macro accepting three
  arguments is written as:\\
  \verb|\somemacro{arg 1}{argument 2}{third arg}|






\item If you want to emphasise a part of a sentence you must use
  \verb|\emph|.  It does not always mean to change to italics font.
  It emphasises depending on the surrounding characters.  If all the
  surrounding characters are normal (it is called roman), then it
  becomes italics, and vice versa.
  
  Therefore, you can have \emph{nested emphasize statements like this,
  where there is an emphasise 
  \emph{inside another}  emphasised section of sentence}.

 
\item Mathematical symbols in running text should be written between
two Dollar symbols \$  \$.  As an example see how this is written:
$\sum_{i=0}^{N} x_i $.  Here $x$ is any math variable, it must not
be written like x (without the dollar symbols). Did you notice the
$difference$ ?


\item Math mode does not mean simply italics, even the spacing between
the letters changes.  Consider the difference between
\emph{different} and $different$.


\end{enumerate}

\end{document}

